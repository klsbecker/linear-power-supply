\section{Conclusão}

O projeto da fonte de alimentação linear com realimentação de tensão demonstrou ser funcional e eficiente, atendendo aos requisitos estabelecidos. Desde o processo de concepção, simulação, montagem até a realização dos testes, cada etapa foi crucial para validar o desempenho e a confiabilidade do circuito.

Inicialmente, os resultados mostraram que a tensão de saída foi ajustada e estabilizada dentro dos limites especificados, mesmo diante de variações de carga. Embora tenham sido observadas discrepâncias entre valores calculados, simulados e medidos, estas foram compreendidas como decorrentes de tolerâncias nos componentes e limitações práticas, como a variação do \(V_{BE}\) dos transistores e dissipação térmica. Apesar disso, o circuito se mostrou robusto e capaz de operar de maneira confiável na faixa projetada.

Os testes de corrente máxima indicaram que o circuito protege eficazmente os componentes e a carga em situações de curto-circuito. Embora a corrente medida seja ligeiramente inferior à calculada, o projeto foi validado por operar dentro de limites seguros e atender à funcionalidade desejada.

Outro aspecto relevante foi a avaliação da regulação de carga e estabilidade, que destacaram o desempenho do circuito na manutenção da tensão de saída em diferentes condições. A regulação foi confirmada com base nos resultados experimentais e simulados, validando o modelo teórico e a eficácia da realimentação de tensão.

O monitor de tensão, configurado para atuar como comparador em janela, também foi testado e demonstrou precisão no acionamento dos LEDs indicativos em pontos de tensão predefinidos. Essa funcionalidade garante maior controle e monitoramento da fonte, agregando valor ao projeto.

Por fim, este trabalho exemplifica a importância de uma abordagem metodológica completa, unindo teoria, simulação e experimentação. O projeto alcançou todos os objetivos iniciais e estabeleceu uma base sólida para futuras melhorias, como a utilização de componentes mais precisos ou a inclusão de funcionalidades adicionais, como proteção térmica ou interfaces digitais para monitoramento.

Este estudo reforça que a integração de conceitos teóricos e práticos é essencial para o desenvolvimento de circuitos confiáveis e eficientes, sendo uma excelente aplicação de conhecimentos em eletrônica e engenharia elétrica.