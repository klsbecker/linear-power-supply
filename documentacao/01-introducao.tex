\section{Introdução}

Este trabalho tem como objetivo aplicar os conhecimentos adquiridos na disciplina de Eletrônica Analógica no desenvolvimento de uma fonte de alimentação linear com realimentação de tensão. O projeto visa consolidar conceitos teóricos e práticos abordados em aula, como regulação de tensão, limitação de corrente, dissipação térmica e controle de estabilidade.

A fonte de alimentação projetada possui uma tensão de saída ajustável entre 3 e 15 V, com uma corrente máxima de 1 A. Para garantir um funcionamento seguro, inclui um circuito de limitação automática de corrente, além de LEDs indicadores que sinalizam o estado de operação em termos de tensão e corrente. Esses recursos não apenas tornam o projeto funcional, mas também permitem a visualização prática dos conceitos teóricos estudados.

O foco do trabalho é o ensino e a aplicação prática, proporcionando aos estudantes uma experiência de aprendizado baseada em problemas reais. A realimentação de tensão foi escolhida para demonstrar na prática a importância do controle e da estabilidade em circuitos eletrônicos, sendo este um tema central na disciplina. Além disso, o projeto reforça habilidades essenciais para futuros engenheiros, como análise de circuitos, dimensionamento de componentes e resolução de problemas de engenharia.

\nocite{boylestad, malvino}
